\documentclass[11pt]{article}
\usepackage[left=0.75in, right=0.75in, top=1in, bottom=1in]{geometry}
\usepackage{hyperref}

\title{\vspace{-2cm}
\textsc{Title of the Experiment}
}

\author{\textbf{Your Name} \\ (Student ID)} 

\date{} %% No need to place date here. Leave this line as it is. 

\begin{document}
\maketitle
\vspace*{-6mm}
\begin{center}
%% Write experiment number and date here.
\textbf{Experiment No.} X \hfill \textbf{Date:} DD-MM-YYYY\\
\end{center}
\hrule

\begin{abstract}
Write the abstract/aim of the experiment in 3-4 sentences. 
\end{abstract}

\section{Theory}
Write the concept and relevance of the experiment in \textbf{maximum} 6-7 
simple sentences. Please avoid long complex sentences while writing. 

\section{Approach}
Write the approach that you followed in the experiment. Necessary 
procedures, figures, or design diagrams can be included here.

\section{Observations}
\begin{enumerate}
\item Observation 1
\item Observation 2
\item Observation 3 \\ 
\vdots
\end{enumerate}

\section{Conclusions}
Write major conclusions here.  

\section{References}
Bibliography here, if any.

\section{Code} 
Place the code here. This section is applicable for 
experiments that require coding. You can use the \textbf{listings} 
package to include your code in your document. Click 
\href{https://en.wikibooks.org/wiki/LaTeX/Source_Code_Listings}{here} 
for more information. 


\end{document}

